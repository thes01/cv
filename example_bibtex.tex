% Začátek preambule.
\documentclass{article}
\usepackage[resetfonts]{cmap}
\usepackage{lmodern}
\usepackage[czech]{babel}
\usepackage[latin2]{inputenc}
\usepackage[T1]{fontenc}
\usepackage[backend=biber,
            style=alphabetic,
            maxnames=5,
            alldates=iso8601]{biblatex} % Seznam literatury bude vygenerován 
                                        % BibLaTeXem.
\addbibresource{biblio.bib} % Seznam literatury bude vygenerován 
                                         % ze zdrojů umístěných v souboru 
                                         % databaze-literatury.bib.
% Konec preambule.

% Začátek dokumentu.
\begin{document}
\section*{Citace s~Bib\TeX em}
    Odkaz na literaturu vytvoříme příkazem \verb|\cite{návěští}|. Výsledek pak 
    vypadá např. takto:~\cite{inbook-full} Odkaz na další 
    knihu~\cite{article-full} je automaticky číslován. K~odkazu můžeme přidat 
    i~nějakou poznámku, např.  číslo strany.~\cite[strana 26]{chicago-manual} 
    Citovat můžeme i~více zdrojů najednou.~\cite{inbook-full, article-full}

\printbibliography[heading=bibintoc] % Vysadí seznam z dokumentu odkazovaných 
                                     % citací.
\end{document}
% Konec dokumentu.